\documentclass[sigconf]{acmart}

\usepackage{hyperref}

\usepackage{endfloat}
\renewcommand{\efloatseparator}{\mbox{}} % no new page between figures

\usepackage{booktabs} % For formal tables

\settopmatter{printacmref=false} % Removes citation information below abstract
\renewcommand\footnotetextcopyrightpermission[1]{} % removes footnote with conference information in first column
\pagestyle{plain} % removes running headers

\begin{document}
\title{Big Data in Financial Services}


\author{Huiyi Chen}
\orcid{hid101}
\affiliation{%
  \institution{Indiana University}
  \date{October 2017}
}
\email{huiychen@indiana.edu}



% The default list of authors is too long for headers}
\renewcommand{\shortauthors}{H. CHEN}


\begin{abstract}
This paper provides an insight of how big data has changed the world of financial services. It will identify some key challenges that businesses might face due to the blooming of big data \LaTeX\ It will also describe some applications of big data in financial services, how to approach to big data in finance in practice, and finally, how to get the right outcome.
\end{abstract}

\keywords{Big Data, Financial Services, Finance, Data}


\maketitle

\section{Introduction}

Big data has been around for a long time now. Many scientists and researchers have put in a lot of hours to investigate how big data can benefit human beings in different aspect. One of the most popular fields that people do research in is finance. That reason why it is so crucial to understand big data in finance is that finance is immersed in our daily life, and the cost of generating data and activities within these financial services has long been high. With the increased capability and reducing cost of advanced technology of big data, the potential of improvement in customer engagement and operating performance has increased, promoting businesses to invest more in the big data. The ability to access, analyze, and manage great volumes of data in the same time with evolving the Information Architecture has been crucial to financial services as they improve their business efficiency and performance.\cite{Stackowiak2015} Big data will be the biggest help these financial services could utilize. According to the article How Big Data Has Changed Finance by Trevir Nath, in the years of 2013 to 2015, 90 percent of all the data worldwide has been created as a result of the start of 2.5 quintillion bytes of data everyday.\cite{Nath2015} With that being said, how do we use the data in financial services?


\section{3 V's of Big Data}
The three V's are the fundamental to big data. They are volume, velocity, and variety. \cite{Nath2015} In the past few years, financial services have not been doing so great due to the increasing competition from both within the field and other higher rate of return investment, financial services are constantly seeking new methods to leverage technology to improve efficiency and profit. Using big data will definitely help financial services to develop a competitive advantage. \cite{Nath2015} Coming back to the three fundamentals to big data, velocity is the speed data must be store and analyzed at. According to New York Stock Exchange, it captured 1 terabyte of information everyday. With that being said, by 2016, there will be estimatedly 18.9 billion network connection, making roughly 2.5 connects per person in the world. Financial services can definitely differentiate themselves by focusing on quickly and efficiently processing trade from their competitors.\cite{Turner2013}


\section{Algorithmic Trading}
While back in the days, people make phone calls or come to the trading center to trade and to manage their banking accounts, technology has made this process a lot efficient for clients. Instead of using the traditional operational systems such as ATMs, call centers, or branches and brokerage units, they could now transfer money using the secure internet and trade stocks with one and another at the speed of a click. With the process becoming so convenient for clients, the financial services side need to speed up their processing time in order to keep up with the clients' needs. They can no longer rely on the traditional enterprise data from these old operational systems. They need to forecast more potential client information from sources such as industry data, trading data, news, and analyst reports from internal or competing banks.\cite{Brett2012} Big data shows its remarkable capability in terms of trading. With the growing capabilities of computers algorithmic trading becomes possible. The automated process enables many computer programs to execute financial trades at the speeds and frequencies that any human trader cannot. Within the mathematical models, algorithmic trading provides trades executed at the best possible prices and timely trade placement, and reduces manual errors due to behavioral factor.\cite{Turner2013}




\section{Key Challenges Businesses face}
Almost all companies in the financial services have business intelligence tools and data warehouse for reporting and analyzing customer behavior to optimizing operations and anticipate clients' needs. In order the utilize big data, companies need to deploy Big Data Management Systems that include data reservoirs featuring Hadoop and NoSQL databases.\cite{Stackowiak2015} With the addition of Big Data systems, these financial services could gain higher levels of insight into data
at a higher speed and enable more effective decision making. Here is where one of the key challenges comes from--financial services need to rethink how they want to model their enterprise.\cite{Stackowiak2015}This renewed focus will need to be fueled in part of the new regulatory requirements. In addition to that, since financial services services are incorporating analytical insights into operational decision processes more heavily than before, statistical modeling becomes more and more important for companies to conduct weave prediction, forecasting, and optimization models to create their enterprise analytic fabric.\cite{Stackowiak2015}Using big data on these modeling platforms requires copies of enterprise data which might violate governance policies around data in enterprise warehouse. Financial institutions need to constantly update their data privacy policies according to the newly created policy and keep up with the big data management systems.
Implementing big data into financial services also means taking ricks of losing and leaking data due to the little experience IT people have with big data. Financial services will need to write up a well-planned practice of big data being utilized and examined in the early stage of the process.\cite{Turner2013} 

\subsection{Risk and Capital Management}
As mentioned above, traditional enterprise architectures have served financial services and banks for many decades. The well established systems ensure data safety for clients. With the established systems, clients assure these institutions to market liquidity, operational risk, and manage accounts for them. In addition, these systems have enabled the financial services to manage  capital and meet regulatory requirements.\cite{Turner2013} Adding big data management systems means the possibility of leaking data to unknown institutions and personal data being used to generate data model for malicious behaviors. How the financial services are going to promise to manage clients' information under the ever-changing big data environment will be a big challenge that these financial services will face.\cite{Turner2013}

\subsection{Wealth Management}
Portfolio managers always need to seek to expand their client portfolio and thus, need to shift through massive amount of data and determine whether a shift in the allocations is warranted. There are two steps that they need to pay attention to when gathering data: (1)they would need to detect potential changes in the conversations about vary companies in the portfolio, and (2)investigate the financial conversations to determine the source of information, both socially and internally. These types of analytical capabilities need to be gained through big data and related topic and solutions.\cite{Brett2012}

\section{Getting on track with big data}
It is granted that big data is powerful in the field of finance. However, it must be carefully used in order to maximize its capabilities. According to IBM's Big Data @ Work survey, most organizations are in the early stage of big data planning and development efforts. Financial markets and banking companies are on par with the global pool of cross-industry counterparts.\cite{Turner2013} One of the most important keys to success in using big data is to gain alignment between the financial services needs and the IT designed architecture and deployment within the company. Methodologies that are based on phased approaches are always the most successful ones. To start, we need to understand the current states and the gaps that we need to fill with the new big data management systems so that we can better understand how to progress and build towards to planned state. We also need to customize the architecture as business needs change overtime.\cite{Brett2012} Data security is also a key component. Financial services gather sensitive data that if it is in the wrong hands, a disaster will happen.With that being said, securing access to the data is critical. Finally, we want to leverage reference architectures, data models and appliance-like configuration where possible. \cite{Turner2013} Having reference architecture will speed up the deployment and reduce potential rick of n complete solutions and severe integration challenges.




\section{Conclusions}
Competition in the financial industry will continue to be increasingly strong due to the growing population and financial needs all over the world. \cite{Turner2013} Both banks and financial market are under pressure of responding better and faster to the clients with reliable and quality information. More and more financial institutions are utilizing data to make better decisions about clients and financial markets.\cite{Stackowiak2015} While big data related policies are not comprehensive and are continuously being updated based on political and economic environment, big data will for sure help these institutions and maximize their ability to utilized existing data to generate the most information. Data-driven decision making is already in use to help financial institutions to build strategies, attract clients, and make better trading decisions. Big data will continue to prove its work in financial industry by making them more efficient and useful in the future. 




\bibliographystyle{ACM-Reference-Format}
\bibliography{report} 

\end{document}
